\documentclass[12pt,space]{ctexart} %ans
\usepackage{TestPaperA4}
%\watermark{60}{6}{14-金融工程-白兔兔} %水印
\begin{document}\zihao{5}
% \juemi% 输出绝密
% \biaoti{七中实验学校初一新生分班(摸底)考试卷}
\fubiaoti{数\quad 学}
% {\heiti 注意事项}:
% \begin{enumerate}[itemsep=-0.3em,topsep=0pt]
% 	\item 答卷前,考生务必将自己的姓名和准考证号填写在答题卡上。
% 	\item 回答选择题时,选出每小题答案后,用铅笔把答题卡对应题目的答案标号涂黑。如需改动,用橡皮擦干净后,再选涂其它答案标号。回答非选择题时,将答案写在答题卡上。写在本试卷上无效。
% 	\item 考试结束后,将本试卷和答题卡一并交回。
% 	      请认真核对监考员在答题卡上所粘贴的条形码上的姓名、准考证号与您本人是否相符。
% \end{enumerate}

%%====================================================================
%%—————————————————————————————正文开始———————————————————————————————
%%====================================================================

\section{填空题(每题2分,共20分)}

\begin{enumerate}[itemsep=0.2em,topsep=0pt,resume]%\setcounter{enumi}{12}
	\item $42$和$63$的最大公因数是\blank{}。
	\item 一个数的$\dfrac{1}{9}$是$8$的一半,这个数是\blank{}。
	\item 《数学百科全书》实行八五折优惠后每套$340$元,原价每套\blank{}元。
	\item 从$169$里连续减去$12$,减了\blank{}次后,结果是12。
	\item 在和式$\dfrac{1}{1\times 2}+\dfrac{1}{2\times 3}+\dfrac{1}{3\times 4}+\ldots+\dfrac{1}{n(n+1)}$中,前六项的和是\blank{}。
	\item 一个长方形的长和宽都增加$6$米,周长就增加\blank{}米。
	\item 一个长方形截去一个角后的图形可能是 \blank{}。
	\item 一个书架上有若干本书,小马每次拿出其中$\dfrac{1}{2}$,再放回一本书,一共这样做了5次,书架上还剩$3$本书,原来书架上有\blank{}本书。
	\item 小亮早晨从家汽车到学校,先下坡后上坡,形成情况如图所示,若返回时上坡、下坡的速度仍保持不变,那么小亮从学校骑车回家用的时间是 \blank{} 分钟。
	      \begin{figure}[ht]
		      \centering
		      \begin{minipage}[b]{0.4\textwidth}
			      \begin{asy}
				      texpreamble("\usepackage{ctex}");
				      unitsize(0.9 cm);
				      draw((0,0)--(5,0),EndArrow(6));
				      label("$x$",(5,0),N);
				      label("时间/分钟",(5.5,0),S);
				      draw((0,0)--(0,5),EndArrow(6));
				      label("$y$",(0,5),W);
				      label("路程/百米",(0,5),E);
				      draw((0,0)--(3,1.5));
				      draw((3,0)--(3,1.5),dashed);
				      draw((0,1.5)--(3,1.5),dashed);
				      label("$18$",(3,0),S);
				      label("$36$",(0,1.5),W);
				      draw((3,1.5)--(4,4));
				      draw((4,0)--(4,4),dashed);
				      draw((0,4)--(4,4),dashed);
				      label("$30$",(4,0),S);
				      label("$96$",(0,4),W);
			      \end{asy}
			      \caption{第9题}
		      \end{minipage}
		      \qquad
		      \begin{minipage}[b]{0.4\textwidth}
			      \begin{asy}
				      unitsize(1 cm);
				      // import math;
				      pair A,B,C,D,E;
				      A=(4,4); B=(0,0); C=(5,0); D=(2.5,0);
				      E=(1.35,1.35);
				      // dot(D);dot(E);dot(F);
				      path p = A..D;
				      path q = C..E;
				      pair F = intersectionpoint(p,q);
				      draw(A--B--C--A--D--E--C);
				      label("$A$",A,E);
				      label("$B$",B,W);
				      label("$C$",C,E);
				      label("$D$",D,S);
				      label("$E$",E,NW);
				      label("$F$",F,E+0.1N);
			      \end{asy}
			      \caption{第10题}
		      \end{minipage}
	      \end{figure}
	\item 如图,$\triangle ABC$中,$BD=DC,AE=2BE,AD$与$CE$相交于点$F$,若$\triangle ABC$的面积为$1$,则$\triangle AED$的面积为 \blank{}。
\end{enumerate}

\section{选择题(每题3分,共21分)}

\begin{enumerate}[itemsep=0.2em,topsep=0pt,resume]
	\item 下列说法正确的是(\qquad).
	      \begin{tasks}(4)
		      \task 最小的质数是$1$
		      \task 奇数是质数
		      \task 合数是偶数
		      \task $0$是自然数
	      \end{tasks}
	\item 小圆半径是大圆半径的$\dfrac{1}{3}$,小圆面积是大圆面积的(\qquad).
	      \begin{tasks}(4)
		      \task $\dfrac{1}{3}$
		      \task $\dfrac{2}{3}$
		      \task $\dfrac{1}{9}$
		      \task $\dfrac{4}{9}$
	      \end{tasks}
	\item 走一段路,若甲用$3$小时,乙用$4$小时,则甲的速度与乙的速度的比为(\qquad).
	      \begin{tasks}(4)
		      \task $3:4$
		      \task $4:3$
		      \task $2:5$
		      \task 与路径有关
	      \end{tasks}
	\item 如果$a>b$($a,b$均为自然数,且$b\ne 0$),那么下列式子中,正确的式子是(\qquad).
	      \begin{tasks}(4)
		      \task $\dfrac{1}{a}>\dfrac{1}{b}$
		      \task $\dfrac{1}{a}<\dfrac{1}{b}$
		      \task $\dfrac{c}{a}>\dfrac{c}{b}$
		      \task $\dfrac{c}{a}<\dfrac{c}{b}$
	      \end{tasks}
	\item 一个最简真分数,分子、分母的积是$60$,这样的最简真分数有(\qquad).
	      \begin{tasks}(4)
		      \task 3个
		      \task 4个
		      \task 5个
		      \task 6个
	      \end{tasks}
	\item 在一种盐水中,盐和水的比是$1:9$,那么这种盐水的含盐率是(\qquad).
	      \begin{tasks}(4)
		      \task $9\%$
		      \task $90\%$
		      \task $1\%$
		      \task $10\%$
	      \end{tasks}
	\item 如图,图中有(\qquad)个三角形。
	      \begin{tasks}(4)
		      \task $5$
		      \task $6$
		      \task $9$
		      \task $10$
	      \end{tasks}
	      \begin{figure}[ht]
		      \centering
		      \begin{asy}
			      unitsize(0.8 cm);
			      pair A,B,C,D,E,F,G;
			      A=(0,4);B=(5,0);C=(7,4);
			      D=(3.5,1.2);E=(5,4);F=(2,2.4);G=(3,4);
			      draw(A--B--C--A);
			      draw(F--G--D--E--B);
			      label("$A$",A,W);
			      label("$B$",B,E);
			      label("$C$",C,E);
			      label("$D$",D,S);
			      label("$E$",E,N);
			      label("$F$",F,S);
			      label("$G$",G,N);
		      \end{asy}
		      \caption{第17题}
	      \end{figure}
	\item 下面的算式是从左到右每四个一行,依次往下按某种规律排列的:
	      \[
		      \begin{array}{cccc}
			      1+1,   & 2+3,   & 3+5,   & 4+7,   \\
			      1+9,   & 2+11,  & 3+13,  & 4+15,  \\
			      1+17,  & 2+19,  & \ldots & \ldots \\
			      \ldots & \ldots & \ldots & \ldots
		      \end{array}
	      \]
	      那么和为$2016$的算式是第(\qquad)个算式。
	      \begin{tasks}(4)
		      \task $1005$
		      \task $1006$
		      \task $1007$
		      \task $1008$
	      \end{tasks}
\end{enumerate}

\section{计算题(第19题每小题3分,第20题每小题4分,共22分}
\begin{enumerate}[itemsep=-0.3em,topsep=0pt,resume]
	\item 计算.
	      \begin{tasks}[label=(\arabic*),label-width=1.5em,item-indent=2em](2)
		      \task $\displaystyle\dfrac{3}{4}-0.6\times\Big(2\dfrac{5}{12}-1.75\Big)$
		      \task $\Big(6\dfrac{1}{2}-3\dfrac{3}{4}\Big)\div\Big(13+11\dfrac{1}{5}\Big)$
	      \end{tasks}
	      \vspace{3.5cm}
	\item 巧算.(写出计算过程)
	      \begin{tasks}[label=(\arabic*),label-width=1.5em,item-indent=2em](2)
		      \task $2015\times\dfrac{2013}{2014}$
		      \task $\Big(\dfrac{1}{15}+\dfrac{2}{17}\Big)\times 15\times 17$
		      \vspace{2.4cm}
		      \task $9.81\times 0.1+0.5\times 98.1+0.048\times 981$
		      \task $1+\dfrac{1}{1+2}+\dfrac{1}{1+2+3}+\ldots +\dfrac{1}{1+2+\ldots +100}$
		      \vspace{3.5cm}
	      \end{tasks}
\end{enumerate}

\section{解答题(共34分)}
\begin{enumerate}[itemsep=-0.3em,topsep=0pt,resume]%
	\item (每题4分,共8分)
	      \begin{tasks}[label=(\arabic*),label-width=1.5em,item-indent=2em]
		      \task 已知$a+b=12,a:b=1:3$,求:$a,b$的值。
		      \vspace{3.5cm}
		      \task 一种运算: $m\triangle n=\dfrac{(m-n)x}{m+n}$,若$4\triangle 3=3$,求$6\triangle 5$.
		      \vspace{3.5cm}
	      \end{tasks}
	\item 一辆货车和一辆客车从甲乙两地同时相对开出,4小时后再距中点48千米处相遇。已知货车是客车速度的$\dfrac{5}{7}$,客车和货车的速度各是多少?甲乙两地相距多少千米?(6分)
	      \vspace{3.5cm}
	\item 如图,正方形$ABCD$与正方形$DEFG$的边长分班为$6$厘米,$2$厘米,求阴影部分面积.(6分)
	      \begin{figure}[ht]
		      \raggedleft
		      \begin{asy}
			      unitsize(0.45 cm);
			      pair A,B,C,D,E,F,G;
			      A=(0,6);B=(0,0);C=(6,0);D=(6,6);E=(8,6);F=(8,8);G=(6,8);
			      draw(A--B--C--D--E--F--G--D--A);
			      draw(B--G);draw(E--G);draw(C--E);
			      path p = A..D;
			      path q = B..G;
			      pair H=intersectionpoint(p,q);
			      draw(C--H);
			      fill(C--H--G--E--cycle,mediumgray);
			      draw(C--G);draw(E--H);
			      label("$A$",A,W);
			      label("$B$",B,W);
			      label("$C$",C,0.1E+0.5S);
			      label("$D$",D,SE);
			      label("$E$",E,0.1E+0.5S);
			      label("$F$",F,NE);
			      label("$G$",G,N);
		      \end{asy}
		      % \caption{第23题}
	      \end{figure}
	      \vspace{1cm}
	\item 在图中,长方形的两边长分别为$2$cm和$4$cm,两个四分之一圆弧的半径也分别为$2$cm和$4$cm,求两个阴影部分的面积差(结果保留$\pi$).(7分)
	      \begin{figure}[ht]
		      \raggedleft
		      \begin{asy}
			      unitsize(1 cm);
			      pair O,A,B,C,D;
			      O=(0,0);A=(2,0);B=(4,0);C=(4,4);D=(2,4);
			      draw(O--A--B--C--D--A);
			      path p=arc(A,r=2,angle1=180,angle2=90);
			      path q=arc(B,r=4,angle1=180,angle2=90);
			      draw(p);draw(q);
			      pair E=intersectionpoint(p,A..D);
			      pair F=intersectionpoint(q,A..D);
			      real ang=degrees(F);
			      path q1=arc(B,r=4,angle1=180,angle2=180-ang);
			      path q2=arc(B,r=4,angle1=180-ang,angle2=90);
			      path pp=buildcycle(p,E..F,q1);
			      fill(pp,mediumgray);
			      fill(q2--C--D--cycle,mediumgray);
			      label("$A$",A,S);
			      label("$B$",B,S);
		      \end{asy}
	      \end{figure}
	\item 某市的出租车因车型不同,收费标准也不同:A型车的起步价10元,3千米后每千米价为1.2元;B型车的起步价8元,3千米后每千米价为1.4元。
	      \begin{tasks}
		      \task 乘坐出租车行多少千米时,这两种出租车的费用一样?(3分)
		      \vspace{4cm}
		      \task 如果你要乘坐A型和B型出租车$x(x>3)$千米,从节省费用的角度,你应该乘坐哪种型号的出租车?(4分)
	      \end{tasks}
\end{enumerate}
\end{document}
