
\documentclass[12pt,space]{ctexart} %ans
\usepackage{TestPaperA4}
\begin{document}\zihao{5}
\fubiaoti{面 \quad 积 \quad 专 \quad 项}

%%====================================================================
%%—————————————————————————————正文开始———————————————————————————————
%%====================================================================

\section{切割法}
\begin{figure}[ht]
	\centering
	\begin{minipage}[b]{0.45\textwidth}
		\begin{asy}
			unitsize(2 cm);
			import patterns;
			add("bengal",hatch(0.8mm,blue));

			pair A,B,C;
			A=(0,0);B=(0.5,sqrt(3)/2);C=(1,0);
			path pa=circle(A,1);
			path pb=circle(B,1);
			path pc=circle(C,1);
			draw(pa);draw(pb);draw(pc);
			dot(A);dot(B);dot(C);
			label("A",A,W+0.2S);
			label("B",B,N);
			label("C",C,E+0.2S);

			pair [] ab=intersectionpoints(pa,pb);
			pair D=ab[0];
			label("D",D,W+0.2N);
			pair [] bc=intersectionpoints(pb,pc);
			pair E=bc[0];
			label("E",E,E);
			pair [] ca=intersectionpoints(pc,pa);
			pair F=ca[1];
			label("F",F,S);

			fill(subpath(pa,2/3,4/3)--subpath(pb,2,8/3)--subpath(pc,2,4/3)--cycle,pattern("bengal"));
			fill(subpath(pa,2/3,0)--subpath(pb,-2/3,0)--subpath(pc,2/3,4/3)--cycle,pattern("bengal"));
			fill(subpath(pa,-2/3,0)--subpath(pb,-2/3,-4/3)--subpath(pc,2,8/3)--cycle,pattern("bengal"));

			// draw(D--E,red);draw(A--C,red);draw(arc(F,1,120,60),red);
		\end{asy}
	\end{minipage}
	\qquad
	\begin{minipage}[b]{0.45\textwidth}
		\begin{asy}
			unitsize(2.5 cm);
			// import graph;
			import patterns;
			add("bengal",hatch(0.8mm,blue));

			// xaxis("",xmin=-1.8,xmax=1.5,EndArrow);
			// yaxis("",ymin=-0.1,ymax=2.3,EndArrow);

			pair O,A,B,C,D;
			O=(0,0);A=(0,1);B=(0,2);D=(sqrt(2),sqrt(2));C=(-sqrt(2),sqrt(2));
			path pa=circle(A,1);
			path po=arc(O,2,135,45);
			draw(pa);draw(po);
			draw(O--C);draw(O--D);
			pair E=intersectionpoint(pa,O..C);
			pair F=intersectionpoint(pa,O..D);
			dot(E);dot(F);dot(A);
			// draw(E--F);
			// draw(subpath(po,2,1), red, MidArrow);
			// draw(subpath(pa,1,0), red, MidArrow);
			fill(subpath(po,0,1)--subpath(pa,1,2)--E--C--cycle,pattern("bengal"));
			fill(subpath(po,2,1)--subpath(pa,1,0)--F..D--cycle,pattern("bengal"));
			fill(subpath(pa,2,3)--O..E--cycle,pattern("bengal"));
			fill(subpath(pa,3,4)--F..O--cycle,pattern("bengal"));
			// draw(E--F--B--E,red);
		\end{asy}
	\end{minipage}
\end{figure}
\vspace{1cm}
\section{重叠法}
\begin{figure}[ht]
	\centering
	\begin{minipage}[b]{0.45\textwidth}
		\begin{asy}
			unitsize(1 cm);
			// import graph;
			import patterns;
			add("bengal",hatch(0.8mm,NW,blue));

			// xaxis("",xmin=-1.8,xmax=1.5,EndArrow);
			// yaxis("",ymin=-0.1,ymax=2.3,EndArrow);
			pair A,B,C;
			A=(4,4);B=(4,0);C=(0,0);
			draw(A--B--C--A);
			path c1=arc((4,2),2,270,90);
			path c2=arc(C,4,45,0);
			draw(c1);draw(c2);
			pair D=intersectionpoint(c1,A..C);
			pair E=intersectionpoint(c2,A..C);
			// dot(D);dot(E);
			fill(subpath(c1,1,2)--A..D--cycle,pattern("bengal"));
			fill(c2--subpath(c1,0,1)--D--E--cycle,pattern("bengal"));
			// draw(D--B--E,red);
		\end{asy}
	\end{minipage}
	\qquad
	\begin{minipage}[b]{0.45\textwidth}
		已知下图圆的直径为$20cm$,阴影部分\RN{1}和\RN{2}的差为$7cm^2$,求$BC$的长度。\\
		\begin{asy}
			unitsize(0.18 cm);
			// import graph;
			import patterns;
			add("bengal",hatch(0.8mm,NW,blue+opacity(0.5)));
			// xaxis("",xmin=-1.8,xmax=1.5,EndArrow);
			// yaxis("",ymin=-0.1,ymax=2.3,EndArrow);
			pair A,B,C,O;
			A=(0,20);B=(0,0);C=(5*pi-0.7);O=(0,10);
			draw(A--B--C--A);
			path c=circle(O,10);
			draw(c);
			pair [] inse=intersectionpoints(c,A..C);
			pair D=inse[1];
			real [] it=times(c,D.x);
			path c1=subpath(c,1,it[1]-4);
			path c2=subpath(c,3,it[1]);
			fill(c1--D--A--cycle,pattern("bengal"));
			fill(c2--D--C--B--cycle,pattern("bengal"));
			label("A",A,N);
			label("B",B,S);
			label("C",C,E);
			label("$\rm I$",(6,14.5));
			label("$\rm II$",(10,2));
			// label("D",D,E);
			// write(it[0],it[1]);
			// draw(c1,red,MidArrow);
		\end{asy}
	\end{minipage}
\end{figure}
\vspace{1cm}
\section{平行线模型}
\begin{figure}[ht]
	\centering
	\begin{minipage}[b]{0.45\textwidth}
		如下图,大正方形的边长为$8cm$,小正方形的边长为$6cm$,求阴影部分$\triangle ADF$的面积。\\
		\begin{asy}
			unitsize(0.4 cm);
			import patterns;
			add("bengal",hatch(0.8mm,NW,blue+opacity(0.5)));

			pair A,B,C,D,E,F,G;
			A=(14,6);B=(14,0);C=(8,0);D=(0,0);E=(0,8);F=(8,8);G=(8,6);
			label("A",A,N);
			label("B",B,S);
			label("C",C,S);
			label("D",D,W);
			label("E",E,W);
			label("F",F,N);
			draw(A--B--C--D--E--F--C);draw(A--G);
			filldraw(A--F--D--A--cycle,pattern("bengal"));
		\end{asy}
	\end{minipage}
	\qquad
	\begin{minipage}[b]{0.45\textwidth}
		如下图,大正方形的边长为$8cm$,小正方形的边长为$6cm$,求阴影部分$\triangle BEG$的面积。\\
		\begin{asy}
			unitsize(0.4 cm);
			import patterns;
			add("bengal",hatch(0.8mm,NW,blue+opacity(0.5)));

			pair A,B,C,D,E,F,G;
			A=(14,6);B=(14,0);C=(8,0);D=(0,0);E=(0,8);F=(8,8);G=(8,6);
			label("A",A,N);
			label("B",B,S);
			label("C",C,S);
			label("D",D,W);
			label("E",E,W);
			label("F",F,N);
			label("G",G,NE);
			draw(A--B--C--D--E--F--C);draw(A--G);
			filldraw(B--G--E--B--cycle,pattern("bengal"));
			// draw(E--C,red);
		\end{asy}
	\end{minipage}
\end{figure}
\vspace{1cm}
\section{比例模型}
\begin{figure}[ht]
	\begin{minipage}[b]{0.4\textwidth}
		在$\triangle ABC$中,$AD=DB,AE=EF=FC$,已知阴影部分$\triangle DEF$的面积为6,求$\triangle ABC$的面积。\\
		\begin{asy}
			unitsize(0.8 cm);
			import patterns;
			import olympiad;
			add("bengal",hatch(0.8mm,NW,blue+opacity(0.5)));

			pair A,B,C,E,F;
			A=(0,0);B=(5,4);C=(6,0);E=(2,0);F=(4,0);
			label("A",A,W);
			label("B",B,E);
			label("C",C,E);
			label("E",E,S);
			label("F",F,S);
			draw(A--B--C--A);
			pair D=midpoint(A..B);
			label("D",D,W);
			draw(C--D--E--F--D);
			fill(D--E--F--D--cycle,pattern("bengal"));
		\end{asy}
	\end{minipage}
	\qquad
	\begin{minipage}[b]{0.45\textwidth}
		下图在$\triangle ABC$中,已知$AE:EB=2:7,AD:DC=7:5$,且$S_{\triangle ABC}=120$,求阴影部分$\triangle BDE$的面积。\\
		\begin{asy}
			unitsize(0.8 cm);
			import patterns;
			import olympiad;
			add("bengal",hatch(0.8mm,NW,blue+opacity(0.5)));

			pair A,B,C;
			A=(0,0);B=(5,4);C=(6,0);
			label("A",A,W);
			label("B",B,E);
			label("C",C,E);
			draw(A--B--C--A);
			pair D=waypoint(A..C,7/12);
			pair E=waypoint(A..B,2/7);
			label("D",D,S);
			label("E",E,W+0.5N);
			// dot(D);dot(E);
			draw(E--D--B);
			fill(E--D--B--E--cycle,pattern("bengal"));

		\end{asy}
	\end{minipage}
\end{figure}
\vspace{1cm}
\section{鸟头模型}
\begin{figure}[ht]
	\begin{minipage}[b]{0.4\textwidth}
		如下图所示,$\triangle ABC$和$\triangle ADE$有相同的顶角,则:
		$$\frac{S_{\triangle ADE}}{S_{\triangle ABC}}=\frac{AD\times AE}{AB\times AC}$$.\\
		\begin{asy}
			unitsize(0.8 cm);
			import patterns;
			import olympiad;
			add("bengal",hatch(0.8mm,NW,blue+opacity(0.5)));

			pair A,B,C,D,E;
			A=(2,4);B=(0,0);C=(5.5,0);
			label("A",A,N);
			label("B",B,W);
			label("C",C,RightSide);
			D=waypoint(A..B,1/3);
			E=waypoint(A..C,2/5);
			draw(A--B--C--A);
			draw(D--E);
			label("D",D,W);
			label("E",E,2*RightSide);
			fill(A--D--E--A--cycle,pattern("bengal"));
		\end{asy}
	\end{minipage}
	\qquad
	\begin{minipage}[b]{0.45\textwidth}
		如下图,$\triangle ABC$和$\triangle ADE$有一对角互补(和为$180^{\degree}$),则:
		$$\frac{S_{\triangle ADE}}{S_{\triangle ABC}}=\frac{AD\times AE}{AB\times AC}$$.
		\begin{asy}
			unitsize(0.8 cm);
			import patterns;
			import olympiad;
			add("bengal",hatch(0.8mm,NW,blue+opacity(0.5)));

			pair A,B,C,D,E;
			D=(3,4.5);B=(0,0);C=(6,0);
			label("D",D,N);
			label("B",B,W);
			label("C",C,RightSide);
			A=waypoint(D..B,3/8);
			E=waypoint(A..C,2/5);
			label("A",A,W);
			label("E",E,2*RightSide);
			draw(D--B--C--A--D--E);
			fill(A--D--E--A--cycle,pattern("bengal"));
		\end{asy}
	\end{minipage}
\end{figure}
\vspace{1cm}
\section{蝴蝶模型}
\begin{figure}[ht]
	\begin{minipage}[b]{0.5\textwidth}
		如下图所示,梯形$ABCD$中,对角线$AC$和$BD$相交于$O$点,则:
		\begin{enumerate}
			\item 两个翅膀面积相等:\\
			      $S_{\triangle AOB}=S_{\triangle COD}$
			\item 翅膀面积之积与头尾面积之积相等:\\
			      $S_{\triangle AOB}\cdot S_{\triangle COD}=S_{\triangle AOD}\cdot S_{\triangle BOC}$
		\end{enumerate}
	\end{minipage}
	\qquad
	\begin{minipage}[b]{0.4\textwidth}
		\begin{asy}
			unitsize(1cm);
			import patterns;
			import olympiad;
			add("bengal",hatch(0.8mm,NW,blue+opacity(0.5)));
			pair A,B,C,D,O;
			A=(1,4);B=(0,0);C=(6,0);D=(4,4);
			draw(A--B--C--D--B--C--A--D);
			O=intersectionpoint(A..C,B..D);
			label("A",A,W);
			label("B",B,W);
			label("C",C,E);
			label("D",D,E);
			label("O",O,S);
			fill(A--O--B--A--cycle,pattern("bengal"));
			fill(D--O--C--D--cycle,pattern("bengal"));
		\end{asy}
	\end{minipage}
\end{figure}
\vspace{1cm}
\section{沙漏模型}
\begin{figure}[ht]
	\begin{minipage}[b]{0.45\textwidth}
		如左图所示,$AB\parallel CD$,则:
		\begin{enumerate}
			\item 结论一: $\dfrac{OA}{OD}=\dfrac{OB}{OC}=\dfrac{AB}{CD}$
			\item 结论二: $\dfrac{S_{\triangle AOB}}{S_{\triangle COD}}=\dfrac{AB^2}{CD^2}$
		\end{enumerate}
	\end{minipage}
	\qquad
	\begin{minipage}[b]{0.45\textwidth}
		\begin{asy}
			unitsize(1cm);
			import patterns;
			import olympiad;
			add("bengal",hatch(0.8mm,NW,blue+opacity(0.5)));
			pair A,B,C,D,O;
			A=(1,4);C=(0,0);D=(6,0);B=(4,4);
			draw(A--B--C--D--A);
			O=intersectionpoint(A..D,B..C);
			label("O",O,1.2S);
			label("A",A,W);
			label("B",B,E);
			label("C",C,W);
			label("D",D,E);
		\end{asy}
	\end{minipage}
\end{figure}
\vspace{1cm}
\section{金字塔模型}
\begin{figure}[ht]
	\begin{minipage}[b]{0.45\textwidth}
		如左图所示,在$\triangle ABC$中,$DE\parallel BC$,则:
		\begin{enumerate}
			\item 结论一: $\dfrac{AD}{AB}=\dfrac{AE}{AC}=\dfrac{DE}{BC}$
			\item 结论二: $\dfrac{AD}{DB}=\dfrac{AE}{EC}$
			\item 结论三: $\dfrac{S_{\triangle ADE}}{S_{\triangle ABC}}=\dfrac{DE^2}{BC^2}$
		\end{enumerate}
	\end{minipage}
	\qquad
	\begin{minipage}[b]{0.45\textwidth}
		\begin{asy}
			unitsize(0.9 cm);
			import patterns;
			import olympiad;
			add("bengal",hatch(0.8mm,NW,blue+opacity(0.5)));
			pair A,B,C,D,E;
			A=(2,5);B=(0,0);C=(7,0);
			draw(A--B--C--A);
			label("A",A,N);
			label("B",B,W);
			label("C",C,RightSide);
			D=waypoint(A..B,3/5);
			E=waypoint(A..C,3/5);
			draw(D--E);
			label("D",D,W);
			label("E",E,2*RightSide);
		\end{asy}
	\end{minipage}
\end{figure}
\vspace{1cm}
\section{风筝模型}
\begin{figure}[ht]
	\begin{minipage}[b]{0.45\textwidth}
		如左图所示,在$\triangle ABC$中,$DE\parallel BC$,则:
		\begin{enumerate}
			\item 结论一(对角相乘,乘积相等):
			      $$S_1 S_3=S_2 S_4$$
			\item 结论二:
			      $$\frac{S_1+S_2}{S_3+S_4}=\frac{AO}{AC}$$
		\end{enumerate}
	\end{minipage}
	\qquad
	\begin{minipage}[b]{0.45\textwidth}
		\begin{asy}
			unitsize(0.9 cm);
			import patterns;
			import olympiad;
			add("bengal",hatch(0.8mm,NW,blue+opacity(0.5)));
			pair A,B,C,D,O;
			A=(2.5,2);B=(0,0);C=(3.5,-4);D=(5.5,0);
			draw(A--B--C--D--A);draw(A--C);draw(B--D);
			O=intersectionpoint(A..C,B..D);
			label("A",A,N);
			label("B",B,W);
			label("C",C,RightSide);
			label("D",D,E);
			label("O",O,SW);
			label("$S_1$",(2,0.7));
			label("$S_2$",(3.5,0.7));
			label("$S_3$",(2,-1));
			label("$S_4$",(3.5,-1));
		\end{asy}
	\end{minipage}
\end{figure}
\vspace{2cm}
\section{燕尾模型}
\begin{figure}[ht]
	\begin{minipage}[b]{0.45\textwidth}
		如左图所示,在$\triangle ABC$中,$D$为边$BC$上任一点,$O$是$AD$上任一点,则:
		\begin{enumerate}
			\item 结论一(内比):
			      \[
				      \frac{S_1}{S_2}=\frac{S_3}{S_4}=\frac{S_1+S_3}{S_2+S_4}=\frac{AO}{OD}
			      \]
			\item 结论二(外比):
			      \[
				      \frac{S_1}{S_3}=\frac{S_2}{S_4}=\frac{S_1+S_2}{S_3+S_4}=\frac{BD}{CD}
			      \]
		\end{enumerate}
	\end{minipage}
	\qquad
	\begin{minipage}[b]{0.45\textwidth}
		\begin{asy}
			unitsize(0.9 cm);
			import patterns;
			import olympiad;
			add("bengal",hatch(0.8mm,NW,blue+opacity(0.5)));
			pair A,B,C,D,O;
			A=(4,5);B=(0,0);C=(6,0);D=(2.5,0);
			draw(A--B--D--A--C--D);
			O=waypoint(A..D,4/7);
			draw(B--O--C);
			label("A",A,N);
			label("B",B,W);
			label("C",C,E);
			label("D",D,S);
			label("O",O,0.7N+1.2E);
			label("$S_1$",(2.8,2.7));
			label("$S_3$",(4.3,2.7));
			label("$S_2$",(2.3,0.8));
			label("$S_4$",(3.6,0.8));
		\end{asy}
	\end{minipage}
\end{figure}
\vspace{1cm}
\section{求解下列各图中阴影部分的面积}
\begin{enumerate}
	\item 如下图,已知小正方形的边长为$6cm$,大正方形边长为$10cm$,求阴影部分三角形的面积。\\
	      \begin{figure}[ht]
		      \raggedleft
		      \begin{asy}
			      unitsize(0.4 cm);
			      import patterns;
			      add("bengal",hatch(0.8mm,blue));
			      pair A,B,C,D,E,F,G,H;
			      A=(0,0);B=(0,6);C=(6,6);D=(6,0);E=(16,0);F=(16,10);G=(6,10);
			      draw(A--B--C--D--A--F--E--D--G--F);
			      Label AD=Label("$6cm$",align=(0,0),position=MidPoint,filltype=Fill(white));
			      draw((0,-0.5)--(6,-0.5),L=AD,arrow=Arrows(),bar=Bars);
			      Label DE=Label("$10cm$",align=(0,0),position=MidPoint,filltype=Fill(white));
			      draw((6,-0.5)--(16,-0.5),L=DE,arrow=Arrows(),bar=Bars);
			      H=intersectionpoint(A..F,C..D);
			      draw(E--H);
			      fill(A--H--E--A--cycle,pattern("bengal"));
		      \end{asy}
	      \end{figure}
	\item 下图的长方形长为$15cm$,宽为$8cm$,已知阴影部分的面积为$68cm^2$,求四边形$EFGO$的面积。\\
	      \begin{figure}[ht]
		      \raggedleft
		      \begin{asy}
			      unitsize(0.48 cm);
			      import patterns;
			      add("bengal",hatch(0.8mm,blue));
			      pair A,B,C,D,E,F,G,O;
			      A=(0,8);B=(0,0);C=(15,0);D=(15,8);F=(4,0);
			      draw(A--B--C--D--A--C);draw(A--F--D--B);
			      label("A",A,W);label("B",B,W);
			      label("C",C,RightSide);label("D",D,RightSide);label("F",F,S);
			      E=intersectionpoint(A..F,B..D);
			      G=intersectionpoint(A..C,D..F);
			      O=intersectionpoint(A..C,B..D);
			      label("E",E,1.2S+0.2W);label("G",G,S);label("O",O,S);
			      fill(A--B--E--A--cycle,pattern("bengal"));
			      fill(A--O--D--A--cycle,pattern("bengal"));
			      fill(C--G--D--C--cycle,pattern("bengal"));
		      \end{asy}
	      \end{figure}
	      \newpage
	\item $\star$已知正方形 的边长为$4cm$,求阴影部分的面积。\\
	      \begin{figure}[ht]
		      \raggedleft
		      \begin{asy}
			      unitsize(1.3 cm);
			      import patterns;
			      add("bengal",hatch(0.8mm,blue));
			      pair A,B,C,D,E;
			      A=(0,0);B=(4,0);C=(4,4);D=(0,4);
			      draw(A--B--C--D--A);
			      Label BC=Label("$4cm$",align=(0.1,0),position=MidPoint,filltype=Fill(white));
			      draw((4.2,0)--(4.2,4),L=BC,arrow=Arrows(),bar=Bars);
			      path c1=arc(B/2,2,180,0);
			      path c2=arc(D,4,-90,0);
			      draw(c1);draw(c2);
			      pair [] i=intersectionpoints(c1,c2);
			      // dot(i[0],blue);dot(i[1],red);
			      real [] it1=times(c1,i[1].x);
			      real [] it2=times(c2,i[1].x);
			      write(i[1]);
			      path c1t=subpath(c1,it1[0],0);
			      path c2t=subpath(c2,0,it2[0]);
			      // draw(c1t,red,MidArrow);
			      // draw(c2t,red,MidArrow);
			      fill(c1t--c2t--cycle,pattern("bengal"));
		      \end{asy}
	      \end{figure}
	\item  $\star$已知正方形 的边长为$4cm$,求阴影部分的面积。\\
	      \begin{figure}[ht]
		      \raggedleft
		      \begin{asy}
			      unitsize(1.3 cm);
			      import patterns;
			      add("bengal",hatch(0.8mm,blue));
			      pair A,B,C,D,O;
			      A=(0,0);B=(4,0);C=(4,4);D=(0,4);O=(2,2);
			      draw(A--B--C--D--A);
			      Label BC=Label("$4cm$",align=(0.1,0),position=MidPoint,filltype=Fill(white));
			      draw((4.2,0)--(4.2,4),L=BC,arrow=Arrows(),bar=Bars);
			      path c0=circle(O,2);
			      path c1=arc(B,4,180,90);
			      path c2=arc(D,4,-90,0);
			      draw(c0);draw(c1);draw(c2);
			      // C0 and C1 intersect at point E & F
			      pair [] ef=intersectionpoints(c0,c1);
			      pair E=ef[0];
			      pair F=ef[1];
			      // get time of E & F at C1
			      real [] e1t=times(c1,E.x);
			      real [] f1t=times(c1,F.x);
			      // define the arc of E & F on C1
			      path c1ef=subpath(c1,e1t[0],f1t[0]);
			      // draw(c1ef,red,MidArrow);
			      real [] e0t=times(c0,E.x);
			      real [] f0t=times(c0,F.x);
			      path c0ef=subpath(c0,4-f0t[0],e0t[0]);
			      // Fil the intersection area of C1 & C0
			      fill(c1ef--c0ef--cycle,pattern("bengal"));
			      // draw(c0ef,red,MidArrow);

			      pair [] gh=intersectionpoints(c0,c2);
			      pair G=gh[0];
			      pair H=gh[1];
			      real [] g2t=times(c2,G.x);
			      real [] h2t=times(c2,H.x);
			      path c2gh=subpath(c2,g2t[0],h2t[0]);
			      // draw(c2gh,red,MidArrow);
			      real [] g0t=times(c0,G.x);
			      real [] h0t=times(c0,H.x);
			      // write(g0t);write(h0t);
			      path c0gh=subpath(c0,4-h0t[0],4+g0t[0]);
			      // draw(c0gh,red,MidArrow);
			      fill(c2gh--c0gh--cycle,pattern("bengal"));
		      \end{asy}
	      \end{figure}
	\item 已知下图的四边形,长和宽分别是$4cm$和$2cm$,求阴影部分的面积。\\
	      \begin{figure}[ht]
		      \raggedleft
		      \begin{asy}
			      unitsize(1.5 cm);
			      import patterns;
			      add("bengal",hatch(0.8mm,blue));
			      pair A,B,C,D;
			      A=(0,0);B=(4,0);C=(4,2);D=(0,2);
			      draw(A--B--C--D--A--C);
			      Label BC=Label("$2cm$",align=(0.1,0),position=MidPoint,filltype=Fill(white));
			      draw((4.1,0)--(4.1,2),L=BC,arrow=Arrows(),bar=Bars);
			      Label AB=Label("$4cm$",align=(0,0),position=MidPoint,filltype=Fill(white));
			      draw((0,-0.2)--(4,-0.2),L=AB,arrow=Arrows(),bar=Bars);
			      path c1=arc((0,1),1,90,-90);
			      path c2=arc((2,2),2,180,360);
			      draw(c1);draw(c2);
			      pair [] i=intersectionpoints(A..C,c1);
			      pair E=i[1];
			      real [] e1=times(c1,E.x);
			      path c1e=subpath(c1,2,e1[1]);
			      fill(c1e--E--A--cycle,pattern("bengal"));
			      real [] e2=times(c2,E.x);
			      path c2e=subpath(c2,e2[0],4);
			      fill(c2e--C--E--cycle,pattern("bengal"));
			      path c1ebar=subpath(c1,0,e1[1]);
			      path c2ebar=subpath(c2,e2[0],1);
			      fill(c1ebar--c2ebar--cycle,pattern("bengal"));
		      \end{asy}
	      \end{figure}
	      \newpage
	\item 已知下图长方形的宽为$4cm$,求阴影部分的面积。\\
	      \begin{figure}[ht]
		      \raggedleft
		      \begin{asy}
			      unitsize(1.1 cm);
			      import patterns;
			      add("bengal",hatch(0.8mm,blue));
			      pair A,B,C,D,O,E,F;
			      O=(0,0);A=(-2*sqrt(2),-2);B=(2*sqrt(2),-2);C=(2*sqrt(2),2);D=(-2*sqrt(2),2);
			      draw(A--B--C--D--A);
			      Label BC=Label("$4cm$",align=(0.1,0),position=MidPoint,filltype=Fill(white));
			      draw((2*sqrt(2)+0.2,-2)--(2*sqrt(2)+0.2,2),L=BC,arrow=Arrows(),bar=Bars);
			      path cb=arc((A+B)/2,2*sqrt(2),180,0);
			      path ct=arc((C+D)/2,2*sqrt(2),180,360);
			      path co=circle(O,2);
			      path ci=circle(O,2*sqrt(2)-2);
			      draw(cb);draw(ct);draw(co);draw(ci);
			      fill(subpath(ci,0,2)--subpath(ci,2,4)--cycle,pattern("bengal"));
			      pair [] i=intersectionpoints(cb,ct);
			      // dot(i[0]);dot(i[1],blue);
			      path cbt=subpath(cb,0.5,1.5);
			      path cot=subpath(co,0,2);
			      fill(cbt--cot--cycle,pattern("bengal"));
			      path ctb=subpath(ct,1.5,2.5);
			      path cob=subpath(co,4,2);
			      fill(ctb--cob--cycle,pattern("bengal"));
			      // draw(cob,red,MidArrow);
		      \end{asy}
	      \end{figure}
	\item 如下图所示的正方形内有一点$E$,且满足$DE\perp CE$,已知$CE=10cm$,求阴影部分$\triangle BCE$的面积。\\
	      \begin{figure}[ht]
		      \raggedleft
		      \begin{asy}
			      unitsize(1.3 cm);
			      import patterns;
			      import olympiad;
			      add("bengal",hatch(0.8mm,blue));
			      pair A,B,C,D,E,F;
			      A=(0,0);B=(4,0);C=(4,4);D=(0,4);F=(2.5,1);
			      draw(A--B--C--D--A);
			      E=foot(D,C,F);
			      dot(E);
			      label("A",A,W);label("B",B,RightSide);
			      label("C",C,RightSide);label("D",D,W);
			      label("E",E,SW);
			      draw(D--E--C);draw(E--B);
			      fill(C--E--B--C--cycle,pattern("bengal"));
			      draw(rightanglemark(C,E,D));
			      // draw(circle((2,4),2));
		      \end{asy}
	      \end{figure}
	\item 如下图所示的扇形,已知扇形的半径为$4cm$,$C,D$分别为$OA,OB$的中点,$E,F$为$\arc{AB}$的三等分点,求阴影部分四边形的面积。\\
	      \begin{figure}[ht]
		      \raggedleft
		      \begin{asy}
			      unitsize(1.2 cm);
			      import patterns;
			      import olympiad;
			      add("bengal",hatch(0.8mm,blue));
			      pair O,A,B,C,D,E,F;
			      O=(0,0);A=(4,0);B=(0,4);C=(2,0);D=(0,2);
			      path c=arc(O,4,90,0);
			      draw(c);draw(A--O--B);draw(C--D);
			      F=waypoint(c,1/3);
			      E=waypoint(c,2/3);
			      draw(C--E--F--D);
			      fill(C--E--F--D--C--cycle,pattern("bengal"));
			      label("O",O,W);label("A",A,RightSide);label("B",B,W);
			      label("C",C,S);label("D",D,W);
			      label("E",E,RightSide);label("F",F,N);
		      \end{asy}
	      \end{figure}
	      \newpage
	\item 下图所示正方形的边长为$4cm$,求阴影部分的面积。\\
	      \begin{figure}[ht]
		      \raggedleft
		      \begin{asy}
			      unitsize( 2.3 cm );
			      import patterns;
			      add("bengal",hatch(0.8mm,blue));
			      pair A,B,C,D;
			      A=(0,0);B=(0,2);C=(2,2);D=(2,0);
			      draw(A--B--C--D--A);
			      path ca=arc(A,2,90,0);
			      path cb=arc(D,2,180,90);
			      path cc=arc(C,2,270,180);
			      path cd=arc(B,2,-90,0);
			      draw(ca);draw(cb);draw(cc);draw(cd);
			      path ca1=subpath(ca,2/3,1/3);
			      path cb1=subpath(cb,2/3,1/3);
			      path cc1=subpath(cc,2/3,1/3);
			      path cd1=subpath(cd,1/3,2/3);
			      fill(ca1--cb1--cc1--cd1--cycle,pattern("bengal"));
			      // draw(cd1,red,MidArrow);
		      \end{asy}
	      \end{figure}
	\item $\star$如下图所示的扇形夹角为$60^{\degree}$,半径为$9cm$,内部两个圆分别与两边相切,且大圆和扇形的弧相切,求大圆与小圆的面积.
	      \begin{figure}[ht]
		      \raggedleft
		      \begin{asy}
			      unitsize(0.6 cm);
			      import patterns;
			      add("bengal",hatch(0.8mm,blue));
			      pair O,P,Q,A,B,C;
			      O=(0,0);P=(sqrt(3),1);Q=(3*sqrt(3),3);
			      A=(9,0);B=(9/2,9*sqrt(3)/2);C=(9*sqrt(3)/2,9/2);
			      path co=arc(O,9,0,60);
			      path cp=circle(P,1);
			      path cq=circle(Q,3);
			      draw(co);draw(A--O--B); //draw(O--C);
			      // draw(cp);draw(cq);
			      filldraw(cp,paleblue);
			      filldraw(cq,paleblue);
		      \end{asy}
	      \end{figure}
	\item 如下图所以,小圆$\odot P$与扇形$OAB$的两边相切,且垂直于扇形边$OA$的小圆的切线与扇形的边$OA$和弧$\arc AB$分别相交于点$C,D$,且已知$CD=10$,求图中阴影部分的面积。
	      \begin{figure}[ht]
		      \raggedleft
		      \begin{asy}
			      unitsize(1 cm);
			      import patterns;
            import olympiad;
			      add("bengal",hatch(0.8mm,blue+opacity(0.5)));
            pair O,A,B,C,D,P,E,F;
            O=(0,0);A=(4,0);B=(0,4);C=(2.4,0);P=(1.2,1.2);
            draw(A--O--B);
            path s=arc(O,4,0,90);
            path c=circle(P,1.2);
            draw(s);draw(c);
            label("O",O,W);label("A",A,RightSide);label("B",B,W);
            label("C",C,S);label("P",P,SE);
            D=intersectionpoint(s,C..(2.4,4));
            draw(C--D);label("D",D,N);
			      draw(rightanglemark(O,C,D));
			      // Label CD=Label("$10$",align=(0,0),position=MidPoint,filltype=Fill(white));
			      // draw(C+(0.3,0)--D+(0.3,0),L=CD,arrow=Arrows(),bar=Bars);
            E=(1.2,0);F=(0,1.2);
            path c1=subpath(c,2,3);
            fill(c1--E--O--F--cycle,pattern("bengal"));
            path c2=subpath(c,-1,2);
            fill(c2--F--B--s--A--E--cycle,pattern("bengal"));
            // draw(c2,red,MidArrow);
		      \end{asy}
	      \end{figure}
        \newpage
        \item 已知下图的正方形边长为$4cm$,求两个弓形所相交的阴影部分的面积。
              \begin{figure}[ht]
                \raggedleft
                \begin{asy}
                  unitsize(1 cm);
                  import patterns;
                   // import olympiad;
                  add("bengal",hatch(0.8mm,blue+opacity(0.5)));
                   pair A,B,C,D,O,E,F,G;
                   A=(0,0);B=(4,0);C=(4,4);D=(0,4);O=(2,2);
                   draw(A--B--C--A--D--B);
                   path sl=arc(A,4,0,90);
                   path sr=arc(B,4,90,180);
                   draw(sl);draw(sr);
                   label("A",A,W);label("B",B,RightSide);
                   label("C",C,RightSide);label("D",D,W);
                   label("O",O,S);
                   path sl1=subpath(sl,2/3,1/2);
                   path sr1=subpath(sr,1.5,1+1/3);
                   // fill(sr1--sl1----O----cycle,pattern="bengal");
                   E=(2*sqrt(2),2*sqrt(2));
                   F=(4-2*sqrt(2),2*sqrt(2));
                   // dot(E);dot(F);
                   fill(sr1--sl1--E--O--F--cycle,pattern("bengal"));
                   // draw(sl1,red,MidArrow);
                   // draw(sr1,red,MidArrow);
                \end{asy}
              \end{figure}
\end{enumerate}
\end{document}
